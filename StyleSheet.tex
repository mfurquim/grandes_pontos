\documentclass[12pt,openright,a4paper,conference,onecolumn,twoside,english,french,spanish]{IEEEtran}

\usepackage{shaded}
\usepackage{multirow}
\usepackage{cmap}	
\usepackage{lmodern}	
\usepackage[T1]{fontenc}	
\usepackage[utf8]{inputenc}		
\usepackage{lastpage}		
%\usepackage{indentfirst}
\usepackage{color}	
\usepackage{graphicx}	
\usepackage{units}
\usepackage[brazilian,hyperpageref]{backref}
\usepackage[alf]{abntex2cite}
\usepackage{bold-extra}
\usepackage{eso-pic}
\usepackage{listings}
\usepackage{color}
\usepackage{amssymb, amsmath, pxfonts} %permite simbolos matemáticos
\usepackage{mathrsfs} %permite uso de fontes para conjuntos
\usepackage[normalem]{ulem} %permite sublinhar palavras
\usepackage{mathrsfs} %permite o uso de letras trabalhadas
\usepackage[top=3cm,left=3cm,right=2cm,bottom=3cm]{geometry} %margens
\usepackage{graphicx} %permite inserir figuras

\begin{document}

Universidade de Brasília -- Faculdade UnB Gama\\Disciplina: Técnicas de
Programação\\Professor: Maurício Serrano\\Período: 2º/2015\\Data de
entrega: 24 de Agosto de 2015\\Atualizado dia: 19 de Outubro de
2015\\Estudantes:

\textbf{11/0014863} - \emph{Kleber B. Moreira}\\\textbf{11/0017561} -
\emph{Mateus M. F. Mendonça}\\\textbf{11/0017692} - \emph{Matheus M.
Nascimento}\\\textbf{11/0064879} - \emph{Nicolas Boarin}

\section{Stylesheet - Grandes Pontos}\label{stylesheet---grandes-pontos}

\subsection{1. Comments}\label{comments}

Comments must obey the following pattern:

1.a) One-line comments must be written using a double slash `//' before
the comment, followed by a space ` ` and a capital letter. Do not finish
it with a period;

\begin{Shaded}
\begin{Highlighting}[]
\CommentTok{// For each disc in discs, draw itself}
\end{Highlighting}
\end{Shaded}

1.b) Comments in blocks must be written between the `/*' and `*/'. For
each new line, there must have an asterisk aligned with the previous
one;

\begin{Shaded}
\begin{Highlighting}[]
\CommentTok{/**}
\CommentTok{ * For each pawn, fetch its sprite and push the object into the array.}
\CommentTok{ * Its y coordinate is changed to be drawn on top of the previous one.}
\CommentTok{ */}
\end{Highlighting}
\end{Shaded}

1.c) Comments which precede methods must be written in blocks explaining
its use and its return:

\begin{Shaded}
\begin{Highlighting}[]
\CommentTok{/**}
\CommentTok{ * Window On Click adjust the mouseClick to gridClick and calls resolveTurn()}
\CommentTok{ * }\KeywordTok{@param}\CommentTok{ }\KeywordTok{mouseClick}
\CommentTok{ */}
\OtherTok{window}\NormalTok{.}\FunctionTok{onclick} \NormalTok{= }\KeywordTok{function}\NormalTok{(mouseClick)\{}

  \CommentTok{// Object mouse containing the coordinate (x,y) of the event}
  \KeywordTok{var} \NormalTok{mousePosition = \{}
    \DataTypeTok{x}\NormalTok{: }\OtherTok{mouseClick}\NormalTok{.}\FunctionTok{pageX}\NormalTok{,}
    \DataTypeTok{y}\NormalTok{: }\OtherTok{mouseClick}\NormalTok{.}\FunctionTok{pageY}
  \NormalTok{\};}
  \CommentTok{//----}
\NormalTok{\}}
\end{Highlighting}
\end{Shaded}

1.d) Comments explaining one line of the code must precede the line;

\begin{Shaded}
\begin{Highlighting}[]
\CommentTok{// Create and Seal object to prevent properties addition}
\KeywordTok{var} \NormalTok{discObject = }\KeywordTok{new} \FunctionTok{GameObject}\NormalTok{(discSprite, positionCoordinates, scale);}
\OtherTok{Object}\NormalTok{.}\FunctionTok{seal}\NormalTok{(discObject);}
\end{Highlighting}
\end{Shaded}

1.e) Comments exceeding 80 characters must be broken into multiple lines
following the block pattern;

\begin{Shaded}
\begin{Highlighting}[]
\CommentTok{/**}
\CommentTok{ * For each pawn, fetch its sprite and push the object into the array.}
\CommentTok{ * Its y coordinate is changed to be drawn on top of the previous one.}
\CommentTok{ */}
\end{Highlighting}
\end{Shaded}

\subsection{2. Types and Names}\label{types-and-names}

2.a) Names of Classes, Attributes, and Methods must follow the CamelCase
pattern, which consists in capitalizing the initial letters of each
word.

2.b) Classes names initiates with capital case.

\begin{Shaded}
\begin{Highlighting}[]
\KeywordTok{var} \NormalTok{pawnObject = }\KeywordTok{new} \FunctionTok{GameObject}\NormalTok{(sprite, positionCoordinates, scale);}
\end{Highlighting}
\end{Shaded}

2.c) Attributes names initiates with lower case.

\begin{Shaded}
\begin{Highlighting}[]
\KeywordTok{var} \NormalTok{isValid = }\KeywordTok{false}\NormalTok{;}
\end{Highlighting}
\end{Shaded}

2.d) Methods names initiates with lower case.

\begin{Shaded}
\begin{Highlighting}[]
\OtherTok{GameObject}\NormalTok{.}\OtherTok{prototype}\NormalTok{.}\FunctionTok{move} \NormalTok{= }\KeywordTok{function}\NormalTok{(positionCoordinates) \{}
    \KeywordTok{this}\NormalTok{.}\OtherTok{_sprite}\NormalTok{.}\FunctionTok{positionCoordinates} \NormalTok{= positionCoordinates;}
\NormalTok{\}}
\end{Highlighting}
\end{Shaded}

2.e) Constants names must be capitalized and use underscore between
names.

\begin{Shaded}
\begin{Highlighting}[]
\CommentTok{// Number of discs' colors}
\KeywordTok{const} \NormalTok{NUMBER_DISC_ROWORS = }\DecValTok{7}\NormalTok{;}
\end{Highlighting}
\end{Shaded}

2.f) Only one variable must be declared per line.

\begin{Shaded}
\begin{Highlighting}[]
\KeywordTok{var} \NormalTok{canvas          = }\KeywordTok{null}\NormalTok{,}
    \NormalTok{context         = }\KeywordTok{null}\NormalTok{,}
    \NormalTok{offsetWidth     = }\DecValTok{20}\NormalTok{,}
    \NormalTok{offsetHeight    = }\DecValTok{20}\NormalTok{;}
\end{Highlighting}
\end{Shaded}

\subsection{3. Strings}\label{strings}

3.a) Strings must be written between double quotes `"';

\begin{Shaded}
\begin{Highlighting}[]
\KeywordTok{const} \NormalTok{GREEN_DISC_NAME = }\StringTok{"Multi/disc_green.png"}\NormalTok{;}
\end{Highlighting}
\end{Shaded}

\subsection{4. Indentation}\label{indentation}

4.a) The indentation must be written by two (2) spaces or an equivalent
tab.

\begin{Shaded}
\begin{Highlighting}[]
\KeywordTok{function} \FunctionTok{validateDisc}\NormalTok{(discCount, discColor) \{}

\NormalTok{--}\KeywordTok{var} \NormalTok{isValid = }\KeywordTok{false}\NormalTok{;}

\NormalTok{--}\CommentTok{// Discs white and black are considered special discs}
\NormalTok{--}\KeywordTok{if} \NormalTok{(discColor === WHITE || discColor === BLACK) \{}
\NormalTok{----isValid = (discCount[discColor] < DISC_SPECIAL_LIMIT)}
\NormalTok{--\} }\KeywordTok{else} \NormalTok{\{}
\NormalTok{----isValid = (discCount[discColor] < DISC_LIMIT)}
\NormalTok{--\}}

\NormalTok{--}\KeywordTok{return} \NormalTok{isValid;}
\NormalTok{\}}
\end{Highlighting}
\end{Shaded}

\subsection{5. Braces}\label{braces}

5.a) Opening braces must be used on the same line of the block
structure, after an empty space;

5.b) Closing braces must be written aligned with the statement which
opened the block;

\begin{Shaded}
\begin{Highlighting}[]
\KeywordTok{function} \FunctionTok{validateDisc}\NormalTok{(discCount, discColor) \{}
\NormalTok{|}
\NormalTok{| }\KeywordTok{var} \NormalTok{isValid = }\KeywordTok{false}\NormalTok{;}
\NormalTok{|}
\NormalTok{| }\CommentTok{// Discs white and black are considered special discs}
\NormalTok{| }\KeywordTok{if} \NormalTok{(discColor === WHITE || discColor === BLACK) \{}
\NormalTok{| | isValid = (discCount[discColor] < DISC_SPECIAL_LIMIT)}
\NormalTok{| \} }\KeywordTok{else} \NormalTok{\{}
\NormalTok{| | isValid = (discCount[discColor] < DISC_LIMIT)}
\NormalTok{| \}}
\NormalTok{|}
\NormalTok{| }\KeywordTok{return} \NormalTok{isValid;}
\NormalTok{\}}
\end{Highlighting}
\end{Shaded}

\subsection{6. Classes}\label{classes}

Classes must be according to the following model:

6.a) There must have a blak line after the class signature.

\begin{Shaded}
\begin{Highlighting}[]
\CommentTok{/**}
\CommentTok{ * The constructor to create a GameObject}
\CommentTok{ *}
\CommentTok{ * }\KeywordTok{@param}
\CommentTok{ * Sprite as defined in Atlas.js}
\CommentTok{ * positionCoordinates (x,y)}
\CommentTok{ * scale (width, height)}
\CommentTok{ */}
\KeywordTok{var} \NormalTok{GameObject = }\KeywordTok{function} \NormalTok{(sprite, positionCoordinates, scale) \{}
    \KeywordTok{this}\NormalTok{.}\FunctionTok{_sprite} \NormalTok{= \{\};}
    \KeywordTok{try} \NormalTok{\{}
        \KeywordTok{this}\NormalTok{.}\OtherTok{_sprite}\NormalTok{.}\FunctionTok{name} \NormalTok{= }\OtherTok{sprite}\NormalTok{.}\FunctionTok{name}\NormalTok{;}
        \KeywordTok{this}\NormalTok{.}\OtherTok{_sprite}\NormalTok{.}\FunctionTok{sourceCoordinates} \NormalTok{= }\OtherTok{sprite}\NormalTok{.}\FunctionTok{sourceCoordinates}\NormalTok{;}
        \KeywordTok{this}\NormalTok{.}\OtherTok{_sprite}\NormalTok{.}\FunctionTok{dimensions} \NormalTok{= }\OtherTok{sprite}\NormalTok{.}\FunctionTok{dimensions}\NormalTok{;}
        \KeywordTok{this}\NormalTok{.}\OtherTok{_sprite}\NormalTok{.}\FunctionTok{positionCoordinates} \NormalTok{= positionCoordinates;}
        \KeywordTok{this}\NormalTok{.}\OtherTok{_sprite}\NormalTok{.}\FunctionTok{scale} \NormalTok{= scale;}
    \NormalTok{\}}
    \KeywordTok{catch} \NormalTok{(errorSprite) \{}
        \FunctionTok{alert}\NormalTok{(errorSprite);}
    \NormalTok{\}}
\NormalTok{\};}
\end{Highlighting}
\end{Shaded}

6.b) Prototypes must be used as inheritance:

\begin{Shaded}
\begin{Highlighting}[]
\KeywordTok{function} \FunctionTok{Person}\NormalTok{( name ) \{}
  \KeywordTok{var} \NormalTok{name = name;}
\NormalTok{\}}

\OtherTok{PhysicalPerson}\NormalTok{.}\FunctionTok{prototype} \NormalTok{= }\KeywordTok{new} \FunctionTok{Person}\NormalTok{();}

\KeywordTok{var} \NormalTok{physicalPerson = }\KeywordTok{new} \FunctionTok{PhysicalPerson}\NormalTok{( }\StringTok{"Nome da Pessoa"} \NormalTok{);}
\OtherTok{console}\NormalTok{.}\FunctionTok{log}\NormalTok{( }\OtherTok{physicalPerson}\NormalTok{.}\FunctionTok{name} \NormalTok{); }\CommentTok{// Nome da Pessoa}
\end{Highlighting}
\end{Shaded}

\subsection{7. Control Structure:
\texttt{if}}\label{control-structure-if}

7.a) There must have a blank space immediately after the keyword if.

7.b) There must have a blank space between the parentheses and the
\texttt{if} condition.

7.c) The comparison operators must have a blank space immediately before
and after.

7.d) If the line exceeds 80 characters, it must be broken into multiple
lines and the condition must be aligned with the previous one.

7.e) The body must be written between braces, even when it is a single
line.

7.f) Blocks of \texttt{Else if} or \texttt{else} must be initiated on
the same line of the closing braces.

7.g) There must have an \texttt{else} block after all \texttt{if} or
\texttt{else if} block.

\begin{Shaded}
\begin{Highlighting}[]
\CommentTok{/**}
\CommentTok{ * Is Inside Board checks whether the click was in the board}
\CommentTok{ * }\KeywordTok{@param}\CommentTok{ }\KeywordTok{gridClick}
\CommentTok{ * }\KeywordTok{@return}\CommentTok{ insideBoard}
\CommentTok{ */}
\KeywordTok{function} \FunctionTok{isInsideBoard}\NormalTok{(gridClick) \{}
  \KeywordTok{var} \NormalTok{insideBoard = }\KeywordTok{false}\NormalTok{;}

  \KeywordTok{if} \NormalTok{(}\OtherTok{gridClick}\NormalTok{.}\FunctionTok{x} \NormalTok{< }\OtherTok{gridConfiguration}\NormalTok{.}\FunctionTok{rows} \NormalTok{&&}
      \OtherTok{gridClick}\NormalTok{.}\FunctionTok{y} \NormalTok{< }\OtherTok{gridConfiguration}\NormalTok{.}\FunctionTok{cols} \NormalTok{&&}
      \OtherTok{gridClick}\NormalTok{.}\FunctionTok{x} \NormalTok{>= }\DecValTok{0} \NormalTok{&&}
      \OtherTok{gridClick}\NormalTok{.}\FunctionTok{y} \NormalTok{>= }\DecValTok{0}\NormalTok{) \{}
    \NormalTok{insideBoard = }\KeywordTok{true}\NormalTok{;}
  \NormalTok{\}}
  \KeywordTok{else} \NormalTok{\{}
    \NormalTok{insideBoard = }\KeywordTok{false}\NormalTok{;}
  \NormalTok{\}}

  \KeywordTok{return} \NormalTok{insideBoard;}
\NormalTok{\}}
\end{Highlighting}
\end{Shaded}

\subsection{8. Control structure:
\texttt{while}}\label{control-structure-while}

8.a) \texttt{While} follows the \texttt{if} pattern:

\begin{Shaded}
\begin{Highlighting}[]
\KeywordTok{while} \NormalTok{( count < numElements ) \{}
  \NormalTok{sumElements = sumElements + }\DecValTok{1}\NormalTok{;}
\NormalTok{\}}
\end{Highlighting}
\end{Shaded}

\subsection{9. Control structure:
\texttt{for}}\label{control-structure-for}

9.a) The \texttt{for} structure follows the \texttt{if} pattern.

9.b) Semicolon must be placed immediately after the assignment and the
comparison, and an empty space after it.

\begin{Shaded}
\begin{Highlighting}[]
\CommentTok{// Fill the board with random discs along the x and y axis (rows and columns)}
\KeywordTok{for} \NormalTok{(}\KeywordTok{var} \NormalTok{discX = }\DecValTok{0}\NormalTok{; discX < NUMBER_DISC_COL; discX++) \{}
  \KeywordTok{for} \NormalTok{(}\KeywordTok{var} \NormalTok{discY = }\DecValTok{0}\NormalTok{; discY < NUMBER_DISC_ROW; discY++) \{}
    \CommentTok{// ---}
    \KeywordTok{var} \NormalTok{positionCoordinates = \{}
        \DataTypeTok{x}\NormalTok{:((discX * }\OtherTok{DISC_DIMENSION}\NormalTok{.}\FunctionTok{HEIGHT}\NormalTok{) + }\OtherTok{BOADR_OFFSET}\NormalTok{.}\FunctionTok{X}\NormalTok{),}
        \DataTypeTok{y}\NormalTok{:((discY * }\OtherTok{DISC_DIMENSION}\NormalTok{.}\FunctionTok{WIDTH}\NormalTok{) + }\OtherTok{BOADR_OFFSET}\NormalTok{.}\FunctionTok{Y}\NormalTok{)}
    \NormalTok{\}}
    \CommentTok{// ---}
  \NormalTok{\}}
\NormalTok{\}}
\end{Highlighting}
\end{Shaded}

\subsection{10. Control structure:
\texttt{switch}}\label{control-structure-switch}

a. Defaults and cases must not be indented;

10.b) Case blocks and default blocks must be indented.

10.c) \texttt{break} must be indented as well.

\begin{Shaded}
\begin{Highlighting}[]
\KeywordTok{function} \FunctionTok{fetchDisc}\NormalTok{(discColor) \{}
  \KeywordTok{var} \NormalTok{discSpriteName = }\StringTok{""}\NormalTok{;}
  \KeywordTok{switch} \NormalTok{(discColor) \{}
    \KeywordTok{case} \DataTypeTok{GREEN}\NormalTok{:}
      \NormalTok{discSpriteName = GREEN_DISC_NAME;}
      \KeywordTok{break}\NormalTok{;}

    \KeywordTok{case} \DataTypeTok{BLUE}\NormalTok{:}
      \NormalTok{discSpriteName = BLUE_DISC_NAME;}
      \KeywordTok{break}\NormalTok{;}

    \KeywordTok{case} \DataTypeTok{RED}\NormalTok{:}
      \NormalTok{discSpriteName = RED_DISC_NAME;}
      \KeywordTok{break}\NormalTok{;}

    \KeywordTok{case} \DataTypeTok{PURPLE}\NormalTok{:}
      \NormalTok{discSpriteName = PURPLE_DISC_NAME;}
      \KeywordTok{break}\NormalTok{;}

    \KeywordTok{case} \DataTypeTok{YELLOW}\NormalTok{:}
      \NormalTok{discSpriteName = YELLOW_DISC_NAME;}
      \KeywordTok{break}\NormalTok{;}

    \KeywordTok{case} \DataTypeTok{WHITE}\NormalTok{:}
      \NormalTok{discSpriteName = WHITE_DISC_NAME;}
      \KeywordTok{break}\NormalTok{;}

    \KeywordTok{case} \DataTypeTok{BLACK}\NormalTok{:}
      \NormalTok{discSpriteName = BLACK_DISC_NAME;}
      \KeywordTok{break}\NormalTok{;}

    \KeywordTok{default}\NormalTok{:}
      \CommentTok{// Should never be reached. There are only seven Discs's colors.}
  \NormalTok{\}}
  \KeywordTok{var} \NormalTok{discSprite = }\OtherTok{ATLAS}\NormalTok{.}\FunctionTok{fetchSprite}\NormalTok{(discSpriteName);}
  \KeywordTok{return} \NormalTok{discSprite;}
\NormalTok{\}}
\end{Highlighting}
\end{Shaded}

\subsection{11. Treatment of Exceptions}\label{treatment-of-exceptions}

11.a) The \texttt{try}, \texttt{catch}, and \texttt{finally} blocks
should be on their own line, after each closing braces.

\begin{verbatim}
try {
  this._sprite.name = sprite.name;
  this._sprite.sourceCoordinates = sprite.sourceCoordinates;
  this._sprite.dimensions = sprite.dimensions;
  this._sprite.positionCoordinates = positionCoordinates;
  this._sprite.scale = scale;
}
catch (errorSprite) {
  alert(errorSprite);
}
\end{verbatim}

\subsection{12. Coding language}\label{coding-language}

The coding language for programming techniques applying is English.
\end{document}
